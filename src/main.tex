%%
%% Author: wangjiong
%% 2018/10/6
%%

% Preamble
\documentclass{ctexart}

% Packages
\usepackage{ctex}
\usepackage{hyperref}
\usepackage[T1]{fontenc}
\usepackage[utf8]{inputenc}

\def\dfn#1{\textbf{#1}}
\def\dfnb#1#2{\textbf{#1(#2)}}

% Document
\begin{document}
    \title{《The mind illuminated》笔记}
    \author{John Wang}
    \maketitle
    \tableofcontents

\part{基本概念}
    \section{词语解释}
    \subsection{觉知}
    \dfn{一般意义上的觉知}:虽然“觉知”(awareness)和“意识”(consciousness)有时被视为同义词,但在一般的使用中觉知比意识具有更普遍和广泛的含义。比如,觉知通常表示生物感知刺激并做出反应的能力。这包括非常低等的生物,比如:蠕虫。而且,人对刺激做出反应不一定要意识到这个刺激。因此,我们将一般意义上的觉知定义为:任何作用于神经系统的可造成立即或稍后的影响的印记或记录。因为这种记录既可能会也可能不会产生我们称之为意识的主观经验,一般意义上的觉知具有以下两种不同的形式:\dfnb{有意识的觉知}{conscious awareness}和\dfnb{无意识的觉知}{nonconscious awareness}。

\end{document}