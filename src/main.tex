%%
%% Author: wangjiong
%% 2018/10/6
%%

% Preamble
\documentclass{ctexart}

% Packages
\usepackage{ctex}
\usepackage{hyperref}
\usepackage[T1]{fontenc}
\usepackage[utf8]{inputenc}

\def\dfn#1{\textbf{#1}}
\def\dfnb#1#2{\textbf{#1}\footnote{#2.}}

% Document
\begin{document}
    \title{《The Mind Illuminated》笔记}
    \author{John Wang}
    \date{}
    \maketitle
    \tableofcontents

\part{基本概念}
    \section{词语解释}

    \subsection{意识和觉知}

    \dfnb{意识}{Consciousness}:意识是指在某刻“知道”某事的主观的、第一人称的经验。意识总是对某事的意识,不存在没有对象的意识。意识的对象包括:通过感官刺激产生的各种影像、声音、味道、气味或身体触觉;以及内部生成的诸如思想、记忆、情绪和享乐感受等精神对象。回忆和陈述的能力完全依赖于意识。然而,无法回忆和陈述一个事件并不意味着意识的缺失。因为大部分意识经验会迅速从记忆中消失。

    \dfnb{一般意义的觉知}{Awareness in the general sense}:虽然“觉知”和“意识”有时被视为同义词,但在一般的使用中觉知比意识具有更普遍和广泛的含义。比如,觉知通常与生物感知刺激并做出反应的能力相关联。这包括非常低等的生物,比如:蠕虫。而且,人对刺激做出反应不一定要意识到这个刺激。因此,我们将一般意义上的觉知定义为:任何作用于神经系统的可造成立即或稍后的影响的印记或记录\footnote{觉知就是一种记录,这种记录可以是被意识到的也可能是未意识到的。这种记录源自于身体——这个意识的无意识的基础。}。因为这种记录既可能会也可能不会产生我们称之为意识的主观经验,一般意义上的觉知具有以下两种不同的形式:\dfn{有意识的觉知}和\dfn{无意识的觉知}。

    \dfnb{有意识的觉知}{Conscious awareness}:我们在某刻主观意识到的一般意义的觉知的内容。它有可能被陈述。

    \dfnb{无意识的觉知}{Non-conscious awareness}:我们主观未意识到的一般意义的觉知的内容,它无法在之后回忆或陈述。它有时被称为\dfnb{潜藏的觉知}{Covert awareness}——无意识地“知道”某事。无意识的觉知包括:\dfn{无法意识的觉知}和\dfn{潜意识的觉知}。

    \dfnb{无法意识的觉知}{Unconscious awareness}:未达到可被意识到的程度的觉知。

    \dfnb{潜意识的觉知}{Subconscious awareness}:有可能被意识到,但在那一刻没有成为意识的对象而未被意识到的觉知。

    \subsection{注意与觉知}


\end{document}